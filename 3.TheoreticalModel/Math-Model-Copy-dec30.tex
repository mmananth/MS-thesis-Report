% !TEX TS-program = pdflatex
% !TEX encoding = UTF-8 Unicode

% This is a simple template for a LaTeX document using the "article" class.
% See "book", "report", "letter" for other types of document.

\documentclass[11pt]{report} % use larger type; default would be 10pt

\usepackage[utf8]{inputenc} % set input encoding (not needed with XeLaTeX)
%\usepackage{amsmath}



%%% Examples of Article customizations
% These packages are optional, depending whether you want the features they provide.
% See the LaTeX Companion or other references for full information.

%%% PAGE DIMENSIONS
\usepackage[margin=0.5in]{geometry} % to change the page dimensions
\geometry{a4paper} % or letterpaper (US) or a5paper or....
% \geometry{margin=2in} % for example, change the margins to 2 inches all round
% \geometry{landscape} % set up the page for landscape
%   read geometry.pdf for detailed page layout information

\usepackage{graphicx} % support the \includegraphics command and options

% \usepackage[parfill]{parskip} % Activate to begin paragraphs with an empty line rather than an indent

%%% PACKAGES
\usepackage{booktabs} % for much better looking tables
\usepackage{array} % for better arrays (eg matrices) in maths
\usepackage{paralist} % very flexible & customisable lists (eg. enumerate/itemize, etc.)
\usepackage{verbatim} % adds environment for commenting out blocks of text & for better verbatim
\usepackage{subfig} % make it possible to include more than one captioned figure/table in a single float
% These packages are all incorporated in the memoir class to one degree or another...

%%% HEADERS & FOOTERS
\usepackage{fancyhdr} % This should be set AFTER setting up the page geometry
\pagestyle{fancy} % options: empty , plain , fancy
\renewcommand{\headrulewidth}{0pt} % customise the layout...
\lhead{}\chead{}\rhead{}
\lfoot{}\cfoot{\thepage}\rfoot{}

%%% SECTION TITLE APPEARANCE
\renewcommand\thesection{\arabic{section}}
\renewcommand\thesubsection{\thesection.\arabic{subsection}}
\usepackage{sectsty}
\allsectionsfont{\sffamily\mdseries\upshape} % (See the fntguide.pdf for font help)
% (This matches ConTeXt defaults)

%%% ToC (table of contents) APPEARANCE
\usepackage[nottoc,notlof,notlot]{tocbibind} % Put the bibliography in the ToC
\usepackage[titles,subfigure]{tocloft} % Alter the style of the Table of Contents
\renewcommand{\cftsecfont}{\rmfamily\mdseries\upshape}
\renewcommand{\cftsecpagefont}{\rmfamily\mdseries\upshape} % No bold!



%%% END Article customizations

%%% The "real" document content comes below...

%###################################################################################

\title{Analysis of Garbage Collector Algorithms in Non-Volatile Memory Devices}
%\author{The Samraksh Company}
%\numberofauthors{4}
%\author{Ananth Mahadevan, Dr. Mukundan Sridharan, Dr. Kenneth W Parker, Dr. Rajiv Ramnath}\\

\author{{Ananth Mahadevan$^\dagger$$\oplus$, Mukundan Sridharan$\oplus$,  Kenneth W Parker$\oplus$, Rajiv Ramnath$^\dagger$}\\
email: {\texttt{\{mahadeva\}}, {\{Ramnath\}}}@cse.ohio-state.edu\\
email: {\texttt{\{Mukundan.Sridharan\}}, {\{Kenneth.Parker\}}}@Samraksh.com\\
$\oplus$ The Samraksh Company \\
$^\dagger$ The Ohio State University \\
}

%\date{} % Activate to display a given date or no date (if empty),
         % otherwise the current date is printed 

\DeclareMathSizes{11}{19}{13}{9}

\begin{document}
\maketitle

\makeatletter
\renewcommand\section{\@startsection {section}{1}{\z@}%
    {-3.5ex \@plus -1ex \@minus -.2ex}%
    {2.3ex \@plus.2ex}%
    {\normalfont\large\bfseries}}
\makeatother

%--------------------------------------------------------------------------------------------------------
\section{Abstract}
	The performance results for five different Garbage Collection (GC) algorithms for Non-Volatile Memory Devices for three access patterns are presented in this report. The access patterns include a long-tailed distribution as well as Uniform distribution. The results indicate that Round-Robin style GC algorithms perform much better in all cases than Generational algorithms. This is counter-intuitive to the existing norms. Invocation of the GC in Flash devices is determined by the fullness of the device. Even at low fullness levels, Generational GCs have very low efficiency. In this paper, we compare the efficiency and the time taken for the individual GCs at fullness levels ranging from 2\% to 98\%.  Existing research looks into using Flash as storage for data and RAM as cache. We analyze the performance of a Flash when it is used in place of a RAM.


\section{Introduction}
	Flash memory is an important non volatile storage media due to its low power consumption and small size. It is increasingly being used as a storage device in embedded devices. Embedded devices are constrained by power and low memory capacity. Hence there is a need to have a very efficient primary storage system that allows fast read and write operations and thereby less power consumption. Flash devices generally have fast read accesses but have very slow write accesses. \\

\begin{center}
   \begin{tabular} {|  l | c | r | }
       \hline
	Read time(ms) & Write time (ms) & Erase time (ms) \\ \hline
	4 & 5 & 6 \\ 
       \hline
   \end{tabular}
\end{center}

There are 2 major Flash devices available today - NAND and NOR.
An important component that has an overhead and affects the performance of the storage system is the Garbage Collector (GC) algorithm. This report quantifies the performance of different GCs against different statistical traffic patterns such as Uniform, Pareto and Bi-Modal. Based on prior experiments, we have observed that the traffic pattern for the data can occur in short bursts or can arrive at regular intervals. This is the reason why we chose to select the afore-mentioned traffic patterns. A simulator for the Flash file system as well as the GC algorithms were coded in Matlab. We compare the performance of five different GC algorithms against three traffic patterns. Our experiments indicate that Round-Robin style algorithms have better efficiency than Generational algorithms.\\

In Solid State Devices such as NAND and NOR based Flash, new data is always written out-of-place. In a Log-Structured File System, this reduces the amount of free space and a Garbage Collector algorithm is invoked which defragments the device by moving all valid data together and erasing the invalid data. This is a critical factor in the performance and life-time of a Flash device. In this paper, we present our work in analyzing five different GC algorithms and compare their performances against various parameters. We also present the theoretical model behind our simulation and explain the reason behind the results we have obtained. A major contribution of our work is to measure the performance of Flash devices against traffic patterns that are generally observed in practice.
The major goals of our work are:
\begin{itemize}
\item To find out if Flash can work as a good primary storage system. 
\item To create performance benchmarks and understand which Garbage Collection algorithm is better. 
\item To create statistical models to test the GC algorithms.
\end{itemize}

We simulated an application which is both equally read and write dominant. We also tested an application which has only writes.

The report is organized as follows: section 3 gives details on related work being conducted currently. Section 4 mentions about the current state-of-the-art in Flash algorithms and section 5 provides details on the Mathematical models behind our simulations. We conclude by outlining our results in section 6, details about our simulator in section 7 and finally talk about future work in this area.

\subsection{Methodology}
The GC and the applications will be implemented in Matlab 2011b. The tests will be run on the Ohio Super Computer center’s cluster called Oakley. The fullness level will range from 2% to 98% and the simulations will run for 100,000 read/write accesses per fullness level. The simulations are done for all three traffic patterns for all 6 GC algorithms.

\subsection*{Implementation details:}
The application generates random records and sizes using the rand() function in Matlab. For Pareto distribution, each record is assigned a weight which decides the usage of a record. After the records are generated, based on a coin toss, it is decided whether the next operation will be a read or a write. Every time the GC is accessed, details such as amount of bytes moved, blocks erased, are captured. These details are then used to plot the required graphs. 

\section{Related Work}

\section{Modelling}

\section{Theoretical model - FIFE Uniform}

Assumptions:\\
Total blocks in a flash device, B = 128\\
The size of a block, BlockSize = 8192\\
The device size is, deviceSize = B * BlockSize\\
The fullness level ranges between 1\% and 98\% and let N be the count of records per fullness level.\\
Let the count of dummy records per block be 1. dummyRec = 1\\
Average number of records per block, 
\begin{equation}R_b = \frac{BlockSize}{AvgRecSize} - dummyRec\end{equation}\\


Total number of accesses is, (Number of blocks * Records per block) - (Number of records)\\
\begin{equation}NumOfAccess = ((B-1) * Rb) -  (N-1)\end{equation}\\
NumOfAccess denotes the number of accesses after which a record in a certain block is accessed again. The equation can also be written as: $((B-1) * Rb) - (B-1) * E(ARB_c)$\\
where $(B-1) * E(ARB_c)$ is the total active records in the flash, which is N.\\
\\

((B-1) * Rb) gives the total records (both active and inactive) in the entire flash and subtracting N removes the active records and leaves only the inactive. Therefore after going through the entire set of active and inactive records, we access a certain active record (hence the reason behind subtracting (N-1)). The above equation gives the best case scenario where N records are spread evenly across (B-1) blocks.\\
\\

The expectation for a record to be hit is $\frac{1}{N}$, which is the pdf for Uniform distribution. Let the probability of missing a record be denoted by $P_m$. 
\begin{equation}Pm = \frac{(N-1)}{N}\end{equation}\\
Let the expectation of active records per block that is going to be cleaned be denoted by $E(ARB_c)$\\
\\

Expectation is then, $E(ARB_c) = (P_m)^{NumOfAccess} * Rb$\\
\\
The intuition behind the equation is that, a record is missed for NumOfAccess times before it is hit. Multiplying by Rb gives the relation for all records in a certain block.\\


The expectation of the average number of records per block can be of the exponential form $1-e^{\lambda . x}$ where x is the fullness of flash (\%). To find $\lambda$ we can equate the exponential function to $R_b$ when the flash is 100\% full. 
The corresponding equation is:
$$1-e^{\lambda . x} = R_b$$
$$\lambda = \frac{log(1 - R_b)}{100}$$\\
where 100 represents the fullness of the flash in percentage.\\
\\
Expectation can therefore be given as: $$E(ARB_c) = (1 - e^{\lambda . x})$$\\
where x is the fullness of flash (\%)
\\

For a Uniform Distribution, let the total accesses to the Flash be: 100,000. Out of this, let 1/2 be writes. Hence totalWrites = 50000\\
\\

The total number of times  Garbage Collector(GC) is called decides the total move cost. During one cycle, when active records are moved from one block (x) to another (y), next invocation of the GC is related to the amount of free space in block y. Hence the relation for number of times GC is invoked is:\\
\begin{equation}NumberGCRuns = \frac{totalWrites}{(R_b - E(ARB_c))}\end{equation}\\
where ${(R_b - E(ARB_c))}$ gives the number of free slots where records can be stored.\\
\\

The fullness of the Flash is given by:
\begin{equation}FlashFullness = \frac{((N * AvgRecSize)}{deviceSize}\end{equation}\\

The expected move cost is:
\begin{equation}E(MoveCost) = E(ARB_c) * AvgRecSize * NumberGCRuns\end{equation}\\

The useful write cost is cost involved in writing data to the Flash without any overhead of moving or erasing records: \begin{equation}UsefulWriteCost = totalWrites * AvgRecSize\end{equation}\\

The actual write cost is the cost of erasing and moving records in order to write a new record:
\begin{equation}ActualWriteCost = UsefulWriteCost + ExpMoveCost\end{equation}\\

Efficiency of a Flash device is then:
\begin{equation}Efficiency = \frac{UsefulWriteCost}{ActWriteCost}\end{equation}\\

%-----------------------------------------------------------------------------------

\section{Theoretical model - FIFE Pareto}
Assumptions:\\
Total blocks in a flash device, B = 128\\
The size of a block, BlockSize = 8192\\
The device size is, deviceSize = B * BlockSize\\
The fullness level ranges between 1\% and 98\% and let N be the count of records per fullness level.\\
Let the count of dummy records per block be 1. dummyRec = 1\\
Average number of records per block, 
\begin{equation}R_b = \frac{BlockSize}{AvgRecSize} - dummyRec\end{equation}\\


Total number of accesses is, (Number of blocks * Records per block) - (Number of records)\\
\begin{equation}NumOfAccess = ((B-1) * Rb) -  (N-1)\end{equation}\\

Minimum possible value of a variable X below which it will be accessed with increased frequency
$$X_m = 0.2$$

Positive parameter for Pareto distribution is: 
$$\alpha = 1.3 $$


Average number of records per block can also be represented as, 
$$1-e^{\lambda . x} = R_b$$
$$\lambda = \frac{log(1 - R_b)}{100}$$\\
where 100 represents the fullness of the flash (\%)\\

The calculated value of $\lambda$ is used for different substitutions of x (flash fullness)  to get the expectation of active records per block that is going to be cleaned.
$$E(ARB_c) = 1-e^{\lambda . x}$$

The expectation for a record to be hit is the mean of a random variable for Pareto distribution, which is given by:\\
$$E(X) = \left\{\frac{\alpha . X_m}{\alpha - 1} \right.  if \alpha > 1$$

Therefore, expectation of a miss for Pareto distribution is:
$$P_m = 1 - \frac{\alpha . X_m}{\alpha - 1} $$

The expectation of active records in the block set to be cleaned can be assumed to be exponentially distributed. Expectation can be given as: $$E(ARB_c) = (1 - e^{\lambda . x})$$\\
where x is the fullness of flash.
\\

%-----------------------------------------------------------------------------------

\section{Theoretical model - 2-Gen Uniform}
Assumptions:\\
Average size of a record = s\\
Size of Gen1 = $F_1$\\
Size of Gen2 = $F_2$\\
\\
Total records that can be accomodated in Gen 1, $R_{gen1} = \frac{F_1}{s} $\\
\\
Total records that can be accomodated in Gen 2, $R_{gen2} = \frac{F_2}{s}$\\
\\
$X_1$ = Expectation of Active Records in Gen 1\\
$X_2$ = Expectation of Active Records in Gen 2\\
\\
Probability of a miss, $P_m = \frac{(N - 1)}{N}$\\
\\
Records moved over to Gen 2 during GC is $\delta_2$\\
Records left over in Gen 1 after GC is $\delta_1 = X_1 - \delta_2$\\
\begin{equation}X_1 = \delta_1 + \delta_2\end{equation}

\begin{equation}X_1 = {P_m}^{A_1} * R_{gen1}\end{equation}

\begin{equation}X_2 = {P_m}^{A_2} * R_{gen2}\end{equation}
\\
Count of accesses between cleans of Gen 1, $A_1 = \frac{F_1}{s} - \delta_1$\\
Count of accesses between cleans of Gen 2, $A_2 = \frac{F_2}{s} - \delta_2$\\
\\
Count of accesses between cleans of Gen 2 is, $A_2 = A_1 * R_{21}$, where $R_{21}$ is factor defined below\\
Ratio of cleans between Gen 1 and 2 is, $R_{21} = \frac{Free Space Remaining in Gen 2}{X_1}$\\
\\
Free Space remaining in Gen 2, $FS_2 = R_{gen2} - X_2$\\
Therefore, $R_{21} = \frac{FS_2}{X_1}$\\

%-----------------------------------------------------------------------------------
\section{Results}
\includegraphics [width=3.5in,height=2.5in] {C:/Ananth/FlashBasedFS/GC-StatisticalModel/Regular-CompactAndClean-GarbageCollector/Theoritical-Plots/Plots-Dec16/FIFE-simulation-theoretical-Dec16/EfficiencyVsFullness.png}
\includegraphics [width=3.5in,height=2.5in] {C:/Ananth/FlashBasedFS/GC-StatisticalModel/Regular-CompactAndClean-GarbageCollector/Theoritical-Plots/Plots-Dec16/1Gen-simulation-theoretical-Dec16/EfficiencyVsFullness.png}\\
\includegraphics [width=3.5in,height=2.5in] {C:/Ananth/FlashBasedFS/GC-StatisticalModel/Regular-CompactAndClean-GarbageCollector/Theoritical-Plots/Plots-Dec16/2Gen-simulation-theoretical-Dec16/EfficiencyVsFullness.png}
\includegraphics [width=3.5in,height=2.5in] {C:/Ananth/FlashBasedFS/GC-StatisticalModel/Regular-CompactAndClean-GarbageCollector/Theoritical-Plots/Plots-Dec16/3Gen-simulation-theoretical-Dec18/EfficiencyVsFullness.png}\\
\includegraphics [width=3.5in,height=2.5in] {C:/Ananth/FlashBasedFS/GC-StatisticalModel/Regular-CompactAndClean-GarbageCollector/Theoritical-Plots/Plots-Dec16/NGen-simulation-theoretical-Dec20/EfficiencyVsFullness.png}

\section{Simulations}

\subsection*{Summary}

%-----------------------------------------------------------------------------------

\begin{thebibliography}{9}
  \bibitem{Budilovsky}
  Evgeny Budilovsky, Sivan Toledo, Aviad Zuck. Prototyping a High-Performance Low-Cost Solid-State Disk. \emph{SYSTOR '11 Proceedings of the 4th Annual International Conference on Systems and Storage}

\end{thebibliography}



\end{document}
